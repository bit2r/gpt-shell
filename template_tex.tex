
% 0. 데이터 ---------------------------------------------

shell-lesson-data

% 0. 그림/이미지 -----------------------------------------
%% 마크다운 이미지 ----------------------------------

@fig-ex-01 (도전과제 질문에 사용되는 파일 시스템)에 나온 디렉토리 구조를 상정한다.

![파일 시스템](images/filesystem-challenge.svg){#fig-ex-01 fig-align="center"}

%% R/파이썬 이미지 ----------------------------------
`/Users`  [@fig-home-directory]에서 Nelle 과학자 컴퓨터 계정과, 랩실 동료 미이라(Mummy)와 늑대인간(Wolfman) 디렉토리를 볼 수 있다.

```{r}
#| label: fig-home-directory
#| echo: false
#| fig-align: "center"
#| fig-width: 203
#| fig-cap: ”홈 디렉토리"

knitr::include_graphics("images/home-directories.svg")
```

% 0. 색인 --------------------------------------
\index{}
\textbf{}

% 1. _extension.yml --------------------------------------

contributes:
  format:
    html:
      theme: cosmo
      css: "_extensions/bit2r/bitPublish/style.css"
    pdf:
      documentclass: book
      classoption: draft

% 1. 예제 --------------------------------------
\label{chap:bitpublish}
\newcounter{exam_num_bitpub}
\setcounter{exam_num_bitpub}{0}

\addtocounter{exam_num_bitpub}{1}
\begin{example}{\ref{chap:bitpublish}.\arabic{exam_num_bitpub}}
\examplelabel{ex1}{\ref{chap:bitpublish}.\arabic{exam_num_bitpub}}

예제는 여기에 ..

\end{example}

% 2. 연습문제 --------------------------------------

\begin{Exercise}\label{Ex2}

\noindent 1.  현재 `projects`라는 디렉토리에 있고, 파일 크기를 사람이 읽기 쉬운 형식으로 긴 목록 형식으로 내용을 표시하려고 합니다. 이를 수행하는 명령어를 작성하십시오.
\begin{tasks}[label=(\arabic*)](1)
\task  ls -lh
\task  ls -l -h
\task  ls -hl
\task  ls -h -l
\task  위의 모든 선택사항
\end{tasks}

\end{Exercise}





