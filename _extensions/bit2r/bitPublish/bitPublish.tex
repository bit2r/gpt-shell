%%==============================================================================
%% load packages
%%==============================================================================
\usepackage{diagbox}                        % 테이블 셀에 대각선 표시를 위해
\usepackage[utf8]{inputenc}
\usepackage{setspace}
\usepackage{tocloft}
\usepackage{makeidx}                        % 찾아보기 (색인) 정의를 위해
\usepackage{parskip}
\usepackage[hangul]{xetexko}
\usepackage{listings}                       % shell script code출력을 위함
\usepackage[framemethod=tikz]{mdframed}
\usepackage[unicode]{hyperref}
\usepackage{multirow}
\usepackage[many]{tcolorbox}
\usepackage{makecell}
\usepackage{environ}
\usepackage[tikz]{bclogo}
\usepackage{tikz}
\usepackage{lastpage}
\usepackage{fontawesome5}
% \usepackage{awesomebox}                     % for 시나리오박스


%%==============================================================================
%% 폰트 정의
%%==============================================================================
%% 라틴 셰리프
% https://github.com/stipub/stixfonts
\setmainfont[ExternalLocation=_extensions/bit2r/bitPublish/fonts/STIXTwoText/]{STIXTwoText-Regular.otf}[%
  Ligatures=TeX,
  BoldFont=STIXTwoText-Bold.otf,
  ItalicFont=STIXTwoText-Italic.otf,
  BoldItalicFont=STIXTwoText-BoldItalic.otf
]

%% 라틴 산셰리프
% https://www.1001fonts.com/nimbus-sans-l-font.html
\setsansfont[ExternalLocation=_extensions/bit2r/bitPublish/fonts/Nimbus Sans L/]{NimbusSanL-Reg.otf}[%
  Ligatures=TeX,
  BoldFont=NimbusSanL-Bol.otf,
  ItalicFont=NimbusSanL-RegIta.otf,
  BoldItalicFont=NimbusSanL-BolIta.otf
]

%% 한국어 셰리프
\setmainhangulfont[ExternalLocation=_extensions/bit2r/bitPublish/fonts/KOPUBWORLD_OTF_FONTS/]{KoPubWorld Batang_Pro Light.otf}[%
  Ligatures=TeX,
  BoldFont=KoPubWorld Batang_Pro Bold.otf,
  ItalicFont=KoPubWorld Batang_Pro Light.otf,
  ItalicFeatures = {FakeSlant = 0.167},
  BoldItalicFont=KoPubWorld Batang_Pro Bold.otf,
  ItalicFeatures = {FakeSlant = 0.167}
]

%% 한국어 산셰리프
\setsanshangulfont[ExternalLocation=_extensions/bit2r/bitPublish/fonts/KOPUBWORLD_OTF_FONTS/]{KoPubWorld Dotum_Pro Light.otf}[%
  Ligatures=TeX,
  BoldFont=KoPubWorld Dotum_Pro Bold.otf,
  ItalicFont=KoPubWorld Dotum_Pro Light.otf,
  ItalicFeatures = {FakeSlant = 0.167},
  BoldItalicFont=KoPubWorld Dotum_Pro Bold.otf,
  ItalicFeatures = {FakeSlant = 0.167}
]

%% 한자
\setmainhanjafont[ExternalLocation=_extensions/bit2r/bitPublish/fonts/KOPUBWORLD_OTF_FONTS/]{KoPubWorld Dotum_Pro Light.otf}[%
  Ligatures=TeX,
  BoldFont=KoPubWorld Dotum_Pro Bold.otf,
  ItalicFont=KoPubWorld Dotum_Pro Light.otf,
  BoldItalicFont=KoPubWorld Dotum_Pro Bold.otf
]

%% 모노스페이스
\setmonofont[ExternalLocation=_extensions/bit2r/bitPublish/fonts/D2Coding/]{D2Coding-Ver1.3.2-20180524.ttf}[%
  Scale=0.95,
  Ligatures=TeX,
  BoldFont=D2CodingBold-Ver1.3.2-20180524.ttf,
  ItalicFont=D2Coding-Ver1.3.2-20180524.ttf,
  ItalicFeatures = {FakeSlant = 0.167},
  BoldItalicFont=D2CodingBold-Ver1.3.2-20180524.ttf,
  BoldItalicFeatures = {FakeSlant = 0.167}
]

%% 수식
\setmathfont[ExternalLocation=_extensions/bit2r/bitPublish/fonts/STIXTwoText/]{STIXTwoMath-Regular.otf}


%% 기호글꼴 명령 - 라틴 문자나 CJK 기호를 어떤 폰트로 식자할 것인가.
\xetexkofontregime{latin}%
  [alphs=latin, puncts=latin, colons=latin, parens=latin, cjksymbols=hangul]
\xetexkofontregime{hangul}%
  [alphs=latin, puncts=latin, colons=latin, parens=latin, cjksymbols=hangul]


%%==============================================================================
%% 장평/자간/줄간격 등
%%==============================================================================
%% 줄간격 정의
\linespread{1.5}


%%==============================================================================
%% 절(section)과 서브절(subsection) 타이틀을 돋움체(sans-serif)로 바꾸기
%%==============================================================================
%% Rmarkdown과 titlesec 패키지가 호환되지 않는 이슈가 있음.
%% 아래 두줄의 명령을 입력하지 않으면 에러가 발생함
%% 문제의 원인:
%% https://stackoverflow.com/questions/40439701/cant-knit-to-pdf-with-custom-styles
%% 문제의 해결
%% https://github.com/rstudio/bookdown/issues/677
\let\paragraph\oldparagraph
\let\subparagraph\oldsubparagraph

\usepackage{titlesec}
\titleformat{\section}
  {\sffamily\selectfont\Large\bfseries}{\thesection}{1em}{}
\titleformat{\subsection}
  {\sffamily\selectfont\large\bfseries}{\thesubsection}{1em}{}


%%==============================================================================
%% 컬러 정의
%%==============================================================================
\definecolor{gray95}{gray}{.95}
\definecolor{gray85}{gray}{.85}
\definecolor{aliceblue}{rgb}{0.94, 0.97, 1.0}
\definecolor{ExerciseColor}{gray}{0.65}              % for example
\definecolor{problemblue}{RGB}{100, 134, 158}        % for 시각화전략
\definecolor{light}{HTML}{E6E6FA}
\definecolor{highlight}{HTML}{800080}
\definecolor{dark}{HTML}{330033}
\definecolor{cornflowerblue}{rgb}{0.39, 0.58, 0.93}  % for Exercise


%%==============================================================================
%% hypersetup
%%==============================================================================
\hypersetup{
    colorlinks,
    citecolor=black,
    filecolor=black,
    linkcolor=black,
    urlcolor=black
}

%%==============================================================================
%% Define code blocks - for single space
%%==============================================================================
%% https://stackoverflow.com/questions/73439371/quarto-pdf-output-code-block-line-spacing
\renewenvironment{Shaded}
    {\begin{snugshade}
    \begin{singlespace}
    \linespread{1}
    }
    {\end{singlespace}
    \end{snugshade}
}


%%==============================================================================
%% backtick과 pipe 기반의 단어 강조 폰트 변경
%%==============================================================================
%% *** Quarto에서는 `(backticks)이 \texttt로 변환됨
%% *** 이 명령을 사용할 경우에는 오리지날 \texttt와의 side effect를 조심해야 함
% start markdown의 `(backticks) 강조 구현 -----
% \newcommand{\backticks}{\setmainfont{KoPubWorld돋움체_Pro}\hangulfontspec{KoPubWorld돋움체_Pro}\selectfont}
% \let\oldtexttt\texttt
% \renewcommand{\texttt}[1]{%
%   {\backticks\oldtexttt{#1{}}}
% }
% end markdown의 `(backticks) 강조 구현 -----

% start markdown의 |(pipe) 강조 구현 for LaTex-----
\newcommand{\MakeShortHighlight}[2][\texttt]{%
  \begingroup\lccode`\~=`#2\lowercase{\endgroup
    \def~##1~{#1{##1}}}%
  \catcode`#2=\active}
\newcommand\DeleteShortHightlight[1]{%
  \catcode`#1=12 }

% \MakeShortHighlight[\colorbox{aliceblue}]\|
% end markdown의 |(pipe) 강조 구현 for LaTex-----


%%==============================================================================
%% 객체 정의
%%==============================================================================
\lstset{
  extendedchars=false,
  basicstyle=\small\ttfamily,
  backgroundcolor=\color{gray85}
}


\surroundwithmdframed[linewidth=0pt,innerleftmargin=5pt,backgroundcolor=gray95,font=\small]{verbatim}

\tcbuselibrary{many}


%%==============================================================================
%% shadequote 정의: 강조하는 문장을 표현하기 위해서
%%==============================================================================
%% https://tex.stackexchange.com/questions/16964/block-quote-with-big-quotation-marks
%% Start shadequote define -----------------------------------------------------
\newfontfamily\quotefont[ExternalLocation=_extensions/bit2r/bitPublish/fonts/KOPUBWORLD_OTF_FONTS/]{KoPubWorld Batang_Pro Light.otf}[%
  Ligatures=TeX
]
\newcommand*\quotesize{40} % if quote size changes, need a way to make shifts relative
% Make commands for the quotes
\newcommand*{\openquote}
   {\tikz[remember picture,overlay,xshift=-4ex,yshift=-2.5ex]
   \node (OQ) {\quotefont\fontsize{\quotesize}{\quotesize}\selectfont``};\kern0pt}

\newcommand*{\closequote}[1]
  {\tikz[remember picture,overlay,xshift=4ex,yshift={#1}]
   \node (CQ) {\quotefont\fontsize{\quotesize}{\quotesize}\selectfont''};}

% select a colour for the shading
\colorlet{shadecolor}{gray85}

\newcommand*\shadedauthorformat{\emph} % define format for the author argument

% Now a command to allow left, right and centre alignment of the author
\newcommand*\authoralign[1]{%
  \if#1l
    \def\authorfill{}\def\quotefill{\hfill}
  \else
    \if#1r
      \def\authorfill{\hfill}\def\quotefill{}
    \else
      \if#1c
        \gdef\authorfill{\hfill}\def\quotefill{\hfill}
      \else\typeout{Invalid option}
      \fi
    \fi
  \fi}
% wrap everything in its own environment which takes one argument (author) and one optional argument
% specifying the alignment [l, r or c]
%
\newenvironment{shadequote}[2][l]%
{\authoralign{#1}
\ifblank{#2}
   {\def\shadequoteauthor{}\def\yshift{-2ex}\def\quotefill{\hfill}}
   {\def\shadequoteauthor{\par\authorfill\shadedauthorformat{#2}}\def\yshift{2ex}}
\begin{snugshade}\begin{quote}\openquote}
{\shadequoteauthor\quotefill\closequote{\yshift}\end{quote}\end{snugshade}}
%% End shadequote define -------------------------------------------------------


%%==============================================================================
%% 예제 프레임 정의
%%==============================================================================
%%------------------------------------------------------------------------------
%% 예제 environment - 예제 프레임 정의
%%------------------------------------------------------------------------------
\newenvironment{example}[1]{%
  \mdfsetup{
    skipabove=20pt,
    skipbelow=10pt,
    innertopmargin=0pt,
    innerbottommargin=4pt,
    leftmargin=-13pt,
    splitbottomskip=0ex,
    splittopskip=0ex,
    topline=false,
    leftline=true,
    bottomline=false,
    rightline=false,
    innerrightmargin=0pt,
    innerlinewidth=2pt,
    font=\normalfont,
    frametitle={\textbf{예제 #1.}},
    linecolor=ExerciseColor,
  }
\begin{mdframed}%
}
{\end{mdframed}}

%%------------------------------------------------------------------------------
%% Custom refference for 예제 environment
%%------------------------------------------------------------------------------
%% https://tex.stackexchange.com/questions/18191/defining-custom-labels
\makeatletter
\newcommand{\examplelabel}[2]{%
   \protected@write \@auxout {}{\string \newlabel {#1}{{#2}{\thepage}{#2}{#1}{}} }%
   \hypertarget{#1}{}
}
\makeatother


%%==============================================================================
%% 타이틀 박스 : titlebox
%%==============================================================================
\definecolor{Cgrapefruit}{HTML}{da4453}
\definecolor{Cbittersweet}{HTML}{e95546}
\definecolor{Csunflower}{HTML}{f6ba59}
\definecolor{Cgrass}{HTML}{8bc163}
\definecolor{Cmint}{HTML}{34bc9d}
\definecolor{Caqua}{HTML}{3bb0d6}
\definecolor{Cbluejeans}{HTML}{4b8ad6}
\definecolor{Clavander}{HTML}{977bd5}
\definecolor{Cpinkrose}{HTML}{d870a9}
\definecolor{Clight}{HTML}{e6e9ed}
\definecolor{Cnight}{HTML}{434a53}
\definecolor{Cgray}{HTML}{aab2bc}

\newcommand{\titlebox}[3]{
    \begin{figure}[h]
        \centering
    \begin{tikzpicture}
        \node[anchor=text,text width=\columnwidth-1.2cm, draw, rounded corners, line width=1pt, fill=#2!15, inner sep=5mm] (big) {\\#3};
        \node[draw, rounded corners, line width=.5pt, fill=#2!40, anchor=west, xshift=5mm] (small) at (big.north west) {\sffamily#1};
    \end{tikzpicture}
    \end{figure}
}


%%==============================================================================
%% 연습문제
%%==============================================================================
%% https://tex.stackexchange.com/questions/369265/math-book-how-to-write-exercise-and-answers

\usepackage{stackengine}
\usepackage{tasks}
% \newlength{\longestlabel}
% \settowidth{\longestlabel}{\bfseries viii.}
% \settasks{label=\roman*., label-format={\bfseries}, label-width=\longestlabel,
% item-indent=0pt, label-offset=2pt, column-sep={10pt}}

\usepackage[lastexercise,answerdelayed]{exercise}
\counterwithin{Exercise}{chapter}
\counterwithin{Answer}{chapter}
\renewcounter{Exercise}[chapter]
\newcommand{\QuestionNB}{\bfseries\arabic{Question}.\ }
\renewcommand{\ExerciseName}{\sffamily연습문제}
\renewcommand{\ExerciseHeader}{\noindent\def\stackalignment{l}% code from https://tex.stackexchange.com/a/195118/101651
    \stackunder[0pt]{\colorbox{cornflowerblue}{\textcolor{white}{\textbf{\sffamily\LARGE\ExerciseHeaderNB\;\large\ExerciseName}}}}{\textcolor{cornflowerblue}{\rule{\linewidth}{2pt}}}\medskip}
\renewcommand{\AnswerName}{연습문제 풀이}
\renewcommand{\AnswerHeader}{\ifthenelse{\boolean{firstanswerofthechapter}}%
    {\bigskip\noindent\textcolor{cornflowerblue}{\textbf{CHAPTER \thechapter}}\newline\newline%
        \noindent\bfseries\emph{\textcolor{cornflowerblue}{\AnswerName\ \ExerciseHeaderNB, page %
                \pageref{\AnswerRef}}}\smallskip}
    {\noindent\bfseries\emph{\textcolor{cornflowerblue}{\AnswerName\ \ExerciseHeaderNB, page \pageref{\AnswerRef}}}\smallskip}}
\setlength{\QuestionIndent}{16pt}


%%==============================================================================
%% infobox 정의
%%==============================================================================
\usepackage{ifthen}

\usetikzlibrary{calc}

\definecolor{information}{RGB}{0,142,234}
\definecolor{caution}{RGB}{249,168,37}
\definecolor{warning}{RGB}{211,47,47}
\definecolor{tip}{RGB}{76,175,80}

\NewEnviron{infobox}[2]
  {\par\medskip\noindent
  \begin{tikzpicture}
    \ifthenelse{\equal{#1}{information}}{\def\infoicon{\faInfoCircle}}{}
    \ifthenelse{\equal{#1}{caution}}{\def\infoicon{\faExclamationTriangle}}{}
    \ifthenelse{\equal{#1}{warning}}{\def\infoicon{\faBomb}}{}
    \ifthenelse{\equal{#1}{tip}}{\def\infoicon{\faLightbulb[regular]}}{}
    \node[inner sep=0pt] (box) {\parbox[t]{.99\textwidth}{%
      \begin{minipage}[t]{.12\textwidth}
        \centering
        \tikz[scale=5]\node[scale=2.0,rotate=0]{\color{#1}\infoicon};
      \end{minipage}%
      \begin{minipage}{.83\textwidth}
      \ifx&#2&%
          {}
      \else
          {\textbf{#2}\par\smallskip}
      \fi
      \BODY
      \end{minipage}\hfill}%
    };
    \draw[#1,line width=3pt]
      ( $ (box.north east) + (-5pt,3pt) $ ) -- ( $ (box.north east) + (0,3pt) $ ) -- ( $ (box.south east) + (0,-3pt) $ ) -- + (-5pt,0);
    \draw[#1,line width=3pt]
      ( $ (box.north west) + (5pt,3pt) $ ) -- ( $ (box.north west) + (0,3pt) $ ) -- ( $ (box.south west) + (0,-3pt) $ ) -- + (5pt,0);
  \end{tikzpicture}\par\medskip%
}



%%==============================================================================
%% 아이디어 박스 정의
%%==============================================================================
\NewEnviron{information}[1]
  {\par\medskip\noindent
  \begin{tikzpicture}
    \node[inner sep=0pt] (box) {\parbox[t]{.99\textwidth}{%
      \begin{minipage}[t]{.12\textwidth}
        \centering
        \tikz[scale=5]\node[scale=1.3,rotate=-10]{\bcinfo};
      \end{minipage}%
      \begin{minipage}{.83\textwidth}
      \textbf{#1}\par\smallskip
      \BODY
      \end{minipage}\hfill}%
    };
    \draw[gray!75!black,line width=3pt]
      ( $ (box.north east) + (-5pt,3pt) $ ) -- ( $ (box.north east) + (0,3pt) $ ) -- ( $ (box.south east) + (0,-3pt) $ ) -- + (-5pt,0);
    \draw[gray!75!black,line width=3pt]
      ( $ (box.north west) + (5pt,3pt) $ ) -- ( $ (box.north west) + (0,3pt) $ ) -- ( $ (box.south west) + (0,-3pt) $ ) -- + (5pt,0);
  \end{tikzpicture}\par\medskip%
}

%%==============================================================================
%% 주의 박스 정의
%%==============================================================================
\NewEnviron{caution}[1]
  {\par\medskip\noindent
  \begin{tikzpicture}
    \node[inner sep=0pt] (box) {\parbox[t]{.99\textwidth}{%
      \begin{minipage}{.12\textwidth}
      \centering\tikz[scale=5]\node[scale=1.3,rotate=-14]{\bcattention};
      \end{minipage}%
      \begin{minipage}{.83\textwidth}
      \textbf{#1}\par\smallskip
      \BODY
      \end{minipage}\hfill}%
    };
    \draw[red!75!black,line width=3pt]
      ( $ (box.north east) + (-5pt,3pt) $ ) -- ( $ (box.north east) + (0,3pt) $ ) -- ( $ (box.south east) + (0,-3pt) $ ) -- + (-5pt,0);
    \draw[red!75!black,line width=3pt]
      ( $ (box.north west) + (5pt,3pt) $ ) -- ( $ (box.north west) + (0,3pt) $ ) -- ( $ (box.south west) + (0,-3pt) $ ) -- + (5pt,0);
  \end{tikzpicture}\par\medskip%
}


%%------------------------------------------------------------------------------
%------ 차례 작성
%%------------------------------------------------------------------------------
% \makeindex

%%------------------------------------------------------------------------------
%------ 각주 from lshort-ko.sty
%%------------------------------------------------------------------------------
%%% translator’s footnote
\NewDocumentCommand \trfnote { +m }
{%
    \footnote{[역주]\ #1}%
}
